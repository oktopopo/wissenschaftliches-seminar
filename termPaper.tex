\documentclass[10pt,a4paper,headinclude,twoside, plainheadsepline, open=right, numbers=noenddot, twocolumn]{article}
%
% ------
% Maketitle metadata
\title{\vspace{-5mm}%
	\fontsize{20pt}{10pt}\selectfont
	\textbf{Methoden zur Messung der Hydratation durch tragbare Photoplethysmographie-Sensoren}
	}	
\vspace{-5mm}\date{}
\author{
	\large
       \begin{minipage}[t]{0.33\linewidth}
         \begin{center}
           	\textsc{Olga Litau}\\[2mm]
                 \normalsize	Matr.Nr: 3156218\\
                 \normalsize
                 \href{mailto:olga1.litau@st.oth-regensburg.de}
                 {olga1.litau@st.oth-regensburg.de}      
         \end{center}
       \end{minipage}        
     }
%
%\addto\captionsgerman{\renewcommand{\figurename}{Fig.}}
%
%%%%%%%%%%%%%%%%%%%%%%%%%%%%%%%%%%%%%%%%%%%%%%%%%%%%%%%%%%%
%
% Literaturverzeichnis
%
% Hier eine von zwei Varianten auswaehlen:
% Nummern oder Buchstaben fuer Referenzen
%\usepackage[backend=biber, style=alphabetic, sorting=nyt]{biblatex}
\usepackage[backend=biber, style=numeric-comp, sorting=none]{biblatex}
%
% Hier werden die Referenzen in einer separaten Datei gespeichert
\addbibresource{termPaper.bib}
%
% WICHTIG: Hier wird nicht BibTeX sondern BibLateX verwendet!!
% Deshalb nicht mit bibtex uebersetzen, sondern mit biber
% Das kann man in jedem Tool wie TexMaker oder TexShop als Option einstellen
%

% Spezielle Einstellungen, insbesondere fuer das Literaturverzeichnis,
% aber auch Packages wie amsmath, Groessenanpassungen etc.
\input{./preferences.tex}
%
% ------
% Header/footer
\usepackage{fancyhdr}
	\pagestyle{fancy}
	\fancyfoot[C]{Wissenschaftliches Seminar WS 2018/19 $\cdot$
          $\cdot$ Prof.~Dr.~Doering}
%
	\fancyhead[RE]{Olga Litau}
	\fancyhead[LO]{Methoden zur Messung der Hydratation durch tragbare Photoplethysmographie
Sensoren}	
	\fancyhead[RO,LE]{\thepage}
%
\begin{document}
\pagenumbering{arabic} % ab jetzt arabische Nummerierung
\twocolumn[
%%%%%%%%%%%%%%%%%%%%%%%%%%%%%%%%%%%%%%%%%%%%%%%%%%%%%%%%%%%%%%%%%%%%%
\maketitle
\tableofcontents % Inhaltsverzeichnis
\vspace{2cm}
\begin{abstract}
\noindent Hierher kommt die Zusammenfassung...
\end{abstract}
\vspace{0.2cm}
]


\section{Einleitung}
\label{einleitung}

 


\section{Methoden zur Messung der Hydratation}
\label{methoden zur messung der hydratation}


\subsection{Klinische Anamnese}
\label{klinische anamnese}


\subsection{Laboruntersuchungen}
\label{laboruntersuchungen}


\subsection{Tragbare Sensoren}
\label{tragbare sensoren}


\section{Tragbare Photoplethysmographie Sensoren}
\label{tragbare photoplethysmographie sensoren}


\subsection{Verwendung von Photopletysmographie zur physiologischen Messung}
\label{verwendung von photopletysmographie zur physiologischen messung}


\subsection{Verarbeitung des PPG-Signals}
\label{verarbeitung des ppg-signals}


\section{Zuverlässigkeit der Messungen mit PPG-Sensoren}
\label{zuverlässigkeit der messungen mit ppg-sensoren}

\section{Vergleich der Verfahren anhand von selbst bestimmten Kriterien}
\label{vergleich der verfahren}

\section{Schlussbetrachtung und Ausblick}
\label{schlussbetrachtung und ausblick}


%
%%%%%%%%%%%%%%%%%%%%%%%%%%%%%%%%%%%%%%%%%%%%%%%%%%%%%%%%%%%%%%%%%%%%%%%%%%%%%%%%%%%%
% Literaturverzeichnis
\printbibliography
% Anhang
%\appendix
%\section{Anhang}
%\label{anhang}
\end{document}

