\documentclass[10pt,a4paper,headinclude,twoside, plainheadsepline, open=right, numbers=noenddot, twocolumn]{article}
%
% ------
% Maketitle metadata
\title{\vspace{-5mm}%
	\fontsize{20pt}{10pt}\selectfont
	\textbf{Methoden zur Messung der Hydratation durch tragbare Photoplethysmographie-Sensoren}
	}	
\vspace{-5mm}\date{}
\author{
	\large
       \begin{minipage}[t]{0.33\linewidth}
         \begin{center}
           	\textsc{Olga Litau}\\[2mm]
                 \normalsize	Matr.Nr: 3156218\\
                 \normalsize
                 \href{mailto:olga1.litau@st.oth-regensburg.de}
                 {olga1.litau@st.oth-regensburg.de}      
         \end{center}
       \end{minipage}        
     }
%
%\addto\captionsgerman{\renewcommand{\figurename}{Fig.}}
%
%%%%%%%%%%%%%%%%%%%%%%%%%%%%%%%%%%%%%%%%%%%%%%%%%%%%%%%%%%%
%
% Literaturverzeichnis
%
% Hier eine von zwei Varianten auswaehlen:
% Nummern oder Buchstaben fuer Referenzen
%\usepackage[backend=biber, style=alphabetic, sorting=nyt]{biblatex}
\usepackage[backend=biber, style=numeric-comp, sorting=none]{biblatex}
\usepackage{booktabs}
%
% Hier werden die Referenzen in einer separaten Datei gespeichert
\addbibresource{termPaper.bib}
%
% WICHTIG: Hier wird nicht BibTeX sondern BibLateX verwendet!!
% Deshalb nicht mit bibtex uebersetzen, sondern mit biber
% Das kann man in jedem Tool wie TexMaker oder TexShop als Option einstellen
%

% Spezielle Einstellungen, insbesondere fuer das Literaturverzeichnis,
% aber auch Packages wie amsmath, Groessenanpassungen etc.
% Allgemeines
%\usepackage[automark]{scrpage2} % Kopf- und Fusszeilen
\usepackage{amsmath} % Mathematik
\usepackage{amsfonts}
\usepackage{amssymb}
\usepackage[utf8]{inputenc} % UTF8-Kodierung fuer Umlaute usw
\usepackage{hyperref} % Internetseiten
\usepackage{multirow} % Tabellen-Zellen ueber mehrere Zeilen
\usepackage{multicol} % mehre Spalten auf eine Seite
\usepackage{tabularx} % Fuer Tabellen mit vorgegeben Groessen
\usepackage[ngerman]{babel}
\usepackage{graphicx} % Bilder
\usepackage{epstopdf} % enable eps graphics
\usepackage{color} % Farben
\usepackage{subfig} % mehrere Abbildungen nebeneinander/uebereinander
% Quellcode
\usepackage{listings} % fuer Formatierung in Quelltexten
\usepackage[font=small,labelfont=bf,figurename=Abb.,tablename=Tab.]{caption}
\usepackage{capt-of}
%%%%%%%%%%%%%%%%%%%%%%%%%%%%%%%%%%%%%%%%%%%%%%%%%%%%%%%%%%%
% ------
% Fonts and typesetting settings
\usepackage[sc]{mathpazo}
\usepackage[T1]{fontenc}
\usepackage{microtype}
% ------
% Page layout
\usepackage[hmarginratio=1:1,top=32mm,columnsep=20pt]{geometry}
\usepackage[font=it]{caption}
\usepackage{paralist}

% ------
% Abstract
\usepackage{abstract}
	\renewcommand{\abstractnamefont}{\normalfont\bfseries}
	\renewcommand{\abstracttextfont}{\normalfont\small\itshape}
%
%
% ------
% Titling (section/subsection)
\usepackage{titlesec}
\titleformat{\section}[block]{\large\scshape\centering}{\thesection.}{1em}{}
%
%%%%%%%%%%%%%%%%%%%%%%%%%%%%%%%%%%%%%%%%%%%%%%%%%%%%%%%%%%%
%
% Einstellungen zum Literaturverzeichnis
%
% Anpassen bei "alphabetic"
\ExecuteBibliographyOptions{%
     maxbibnames=99,   % Alle Autoren (kein et al.)
     maxcitenames=1,   % Kuerzel nur aus 1. Autor im Text
     maxalphanames=1,  % nur 1. Autor in der Abkuerzung
     backref=false,    % keine Ruueckverweise auf Zitatseiten
     firstinits=true,  % Vornamen abkuerzen
     isbn=false,       % ISBN ausblenden
     doi=false,        % DOI ausblenden
   }
\renewcommand*{\labelalphaothers}{} % alpha label ohne +
%
\renewbibmacro*{volume+number+eid}{%
     \setunit{\space}\printfield{volume}%
     \iffieldundef{number}{}{%
      \printtext[parens]{\printfield{number}}}%
     \setunit{\addcomma\space}\printfield{eid}}
%
% no word 'pages' for articles in the bibliography (print as is)
\DeclareFieldFormat[article, inproceedings, incollection, unpublished]{pages}{#1} 
% no quotes for article titles (print as is)
\DeclareFieldFormat[article, inproceedings, incollection, online, unpublished]{title}{#1} 
%
\renewbibmacro*{date}{\printdate}
\renewbibmacro*{issue+date}{\usebibmacro{issue}}
\renewbibmacro*{publisher+location+date}{\printlist{publisher}}
%
   \setcounter{biburlnumpenalty}{9000}
   \setcounter{biburlucpenalty}{9000}
   \setcounter{biburllcpenalty}{9999}
%
% "In:" removed for articles; issue/date macros added after note+pages macro
\DeclareBibliographyDriver{article}{%
  \usebibmacro{bibindex}%
  \usebibmacro{begentry}%
  \usebibmacro{author/translator+others}%
  \setunit{\labelnamepunct}\newblock%
  \usebibmacro{title}%
  \newunit%
  \printlist{language}%
  \newunit\newblock%
  \usebibmacro{byauthor}%
  \newunit\newblock%
  \usebibmacro{bytranslator+others}%
  \newunit\newblock%
  \printfield{version}%
  \newunit\newblock%
  \usebibmacro{journal+issuetitle}%
  \newunit%
  \usebibmacro{byeditor+others}%
  \newunit%
  \usebibmacro{note+pages}%
  \setunit{\addcomma\addspace}%
  \usebibmacro{date}%
  \usebibmacro{finentry}}
%
%
\DeclareBibliographyDriver{inproceedings}{%
    \usebibmacro{begentry}%
    \printnames{author}%
    \setunit{\labelnamepunct}\newblock%
    \printfield{title}%
    \setunit{\labelnamepunct}%
	\usebibmacro{in:}%    
    \newblock%
    \ifnameundef{editor}%
    {%
    		\setunit{\adddot\space}%
    		\newunit%
    }%
    {%
     	\setunit{\addspace}%
     	\printnames[byeditor]{editor}%
     	\clearname{editor}%
     	\setunit{\space}%
     	\printtext[parens]{Hrsg.}%
     	\setunit{\addcolon\space}%
     	\newunit%
     }%
	\printfield{booktitle}%
	\setunit{\addcomma\space}%
	\printfield{pages}%
	\setunit{\addcomma\space}%
    \usebibmacro{date}%
    \usebibmacro{finentry}
}

\DeclareBibliographyDriver{book}{%
  \usebibmacro{bibindex}%
  \usebibmacro{begentry}%
  \usebibmacro{author/editor+others/translator+others}%
  \setunit{\labelnamepunct}\newblock
  \usebibmacro{maintitle+title}%
  \newunit
  \printlist{language}%
  \newunit\newblock
  \usebibmacro{byauthor}%
  \newunit\newblock
  \usebibmacro{byeditor+others}%
  \newunit\newblock
  \printfield{edition}%
  \newunit
  \iffieldundef{maintitle}
    {\printfield{volume}%
     \printfield{part}}
    {}%
  \newunit
  \printfield{volumes}%
  \newunit\newblock
  \usebibmacro{series+number}%
  \newunit\newblock
  \printfield{note}%
  \newunit\newblock
  \usebibmacro{publisher+location+date}%
  \newunit\newblock
  \usebibmacro{chapter+pages}%
  \newunit
  \printfield{pagetotal}%
  \newunit\newblock
  \usebibmacro{doi+eprint+url}%
  \newunit\newblock
  \usebibmacro{addendum+pubstate}%
  \setunit{\bibpagerefpunct}\newblock
  \usebibmacro{pageref}%
  \setunit{\addcomma\space}
  \usebibmacro{date}
  \usebibmacro{finentry}}
%  
%
 \DeclareBibliographyDriver{online}{%
   \usebibmacro{bibindex}%
   \usebibmacro{begentry}%
   \ifnameundef{author}
    {\printtext{Autor unbekannt}}
    {
		\usebibmacro{author/editor+others/translator+others}%    
    }%
   \setunit{\labelnamepunct}\newblock
   \usebibmacro{title}%
   \newunitpunct
   \usebibmacro{url+urldate}%
   %\usebibmacro{addendum+pubstate}%
   \usebibmacro{finentry}}  
%%%%%%%%%%%%%%%%%%%%%%%%%%%%%%%%%%%%%%%%%%%%%%%%%%%%%%%%%%%

% Eigene Befehle %%%%%%%%%%%%%%%%%%%%%%%%%%%%%%%%%%%%%%%%%%%%%%%%%
% Matrix
\newcommand{\mat}[1]{
      {\textbf{#1}}
}
\newcommand{\todo}[1]{
      {\colorbox{red}{ TODO: #1 }}
}
\newcommand{\todotext}[1]{
      {\color{red} TODO: #1} \normalfont
}
\newcommand{\info}[1]{
      {\colorbox{blue}{ (INFO: #1)}}
}
\newcommand{\code}[1]{
      {\ttfamily{#1}}
}

%%%%%%%%%%%%%%%%%%%%%%%%%%%%%%%%%%%%%%%%%%%%%%%%%%%%%%
% Groessenanpassungen
%
\setlength{\unitlength}{1cm}
\setlength{\oddsidemargin}{0.3cm}
\setlength{\evensidemargin}{0.3cm}
\setlength{\textwidth}{15.5cm}
\setlength{\topmargin}{-1.2cm}
\setlength{\textheight}{23cm}
\columnsep 0.5cm

%
% ------
% Header/footer
\usepackage{fancyhdr}
	\pagestyle{fancy}
	\fancyfoot[C]{Wissenschaftliches Seminar WS 2018/19 $\cdot$
          $\cdot$ Prof.~Dr.~Doering}
%
	\fancyhead[RE]{Olga Litau}
	\fancyhead[LO]{Methoden zur Messung der Hydratation durch tragbare Photoplethysmographie
Sensoren}	
	\fancyhead[RO,LE]{\thepage}
%
\begin{document}
\pagenumbering{arabic} % ab jetzt arabische Nummerierung
\twocolumn[
%%%%%%%%%%%%%%%%%%%%%%%%%%%%%%%%%%%%%%%%%%%%%%%%%%%%%%%%%%%%%%%%%%%%%
\maketitle
\tableofcontents % Inhaltsverzeichnis
\vspace{2cm}
\begin{abstract}
\noindent Hierher kommt die Zusammenfassung...
\end{abstract}
\vspace{0.2cm}
]


\section{Einleitung}
\label{einleitung}
Dehydratation bezeichnet eine übermäßige Abnahme des Körperwassers, die eine normale tägliche Schwankung überschreitet. 
Mit 50-70\% der Gesamtkörpermasse ist Wasser der chemische Hauptbestandteil des menschlichen Körpers.
Für einen durchschnittlichen jungen Mann mit 70kg Körpergewicht bedeutet das ein Gesamtkörperwasser von 42l \cite{sawka2015hypohydration}.
5-10\% des Gesamtkörperwassers werden täglich umgesetzt \cite{raman2004american}.
Ein Körperwasserdefizit entsteht wenn die Flüssigkeitsaufnahme durch Getränke und Lebensmittel geringer ist als die Ausscheidung durch Urin, Schweiß, Kot und Verdunstung aus den Atemwegen und der Haut \cite{garret2018engineering}.
Der Begriff "Dehydratation" wird häufig als Synonym für hypertone Dehydratation verwendet.
Unter hypertoner Dehydratation versteht man einen Verlust von Wasser ohne entsprechenden Salzverlust. 
Verursacht werden kann eine hypertone Dehydratation durch unzureichende Flüssigkeitsaufnahme, Schwitzen oder Erbrechen.
Diese Form der Dehydratation ist zu unterscheiden von isotoner Dehydratation und hypotoner Dehydratation.
Isotone Dehydratation kennzeichnet einen Verlust von Wasser und Salz-Ionen im gleichen Verhältnis.
Eine isotone Dehydratation tritt beispielsweise als Folge von Durchfall auf.
Hypotone Dehydratation entsteht, wenn im Verhältnis zum Wasserverlust zu viel Salz ausgeschieden wird.
Bei der Verwendung eines Diuretikums, das die vermehrte Ausschwemmung von Urin aus dem menschlichen Körper bewirkt, kann eine hypotone Dehydratation verursacht werden. 
Bei einer Messung ist es von Vorteil zwischen den Arten der Dehydratation differenzieren zu können, um eine effektive Behandlung zu ermöglichen \cite{garret2018engineering}.

Die Methoden zur Messung der Dehydratation sind unterschiedlich und reichen von etablierten Methoden wie der klinischen Anamnese und Untersuchungen im Labor bis hin zu neu aufkommenden Methoden wie tragbaren Sensoren.
Kapitel \ref{methoden zur messung der hydratation} wird einen Überblick über die verschiedenen Methoden geben. 

Vor kurzem wurde von \textbf{Saryadevara et al.} die Verwendung von tragbaren Photoplethysmographie-Sensoren (PPG) zur objektiven Quantifizierung des Hydratationsstatus vorgestellt \cite{suryadevara2015towards}.
Kapitel \ref{tragbare photoplethysmographie sensoren} richtet den Fokus auf tragbare PPG-Sensoren und erläutert sowohl die Verwendung von Photopletysmographie zur physiologischen Messung als auch die Verarbeitung eines PPG-Signals.

In Kapitel \ref{vergleich der verfahren} wird die Zuverlässigkeit von PPG-Methoden anhand der verfügbaren Daten im Vergleich zu klinischen Methoden bewertet.

Abschließend erfolgt in Kapitel \ref{schlussbetrachtung und ausblick} eine Schlussbetrachtung mit Ausblick. 



\section{Methoden zur Messung der Hydratation}
\label{methoden zur messung der hydratation}

\begin{table*}[ht]
%\begingroup
\centering
\begin{tabular*}{15cm}{p{4.5cm}p{4.5cm}p{5cm}}
\toprule
\multicolumn{2}{c}{\textbf{Etablierte Methoden}} & \multicolumn{1}{l}{\textbf{Aufkommende Methoden}} \\
 \midrule
Klinische Anamnese & Laboruntersuchungen & \\
\cmidrule(r){1-1}\cmidrule(lr){2-2}
Durst & Blutanalyse  & \textit{Visuelle/optische Messungen} \\
Hautturgor & Urinanalyse & Elektromagnetische Messungen \\
Blutdruck & Messung des Gewichts & Chemische Verfahren \\
Herzfrequenz & Bioimpedanzanalyse & Akustische Verfahren \\
\bottomrule
\end{tabular*}
\caption{Überblick über die Methoden zur Messung der Dehydratation. Die Messung durch PPG-Sensoren ist eine aufkommende Methode und kann den visuellen/optischen Methoden zugeordnet werden.}
\label{methodenübersicht}
\end{table*}
%\endgroup


Um eine Grundlage für die Bewertung von PPG-Sensoren zu schaffen, werden in diesem Kapitel sowohl weit verbreitete als auch aufkommende Methoden (vergleiche Tab.~\ref{methodenübersicht}) vorgestellt.
Für die klinische Verwendung sollten Dehydratationsmessungen in der Lage sein, Schwankungen innerhalb von 3\% des Gesamtkörperwassers oder etwa
2\% des Körpergewichts für eine durchschnittliche Person zu erfassen.
Jedoch können bestimmte Risikogruppen wie z.B. ältere Erwachsene bereits Symptome mit weniger ausgeprägten Änderungen der Dehydratation aufweisen \cite{garret2018engineering}.
Die derzeit etablierten Methoden bestehen aus klinischer Anamnese und verschiedenen labortechnischen Untersuchungen.
Zu den aufkommenden Methoden zählen u.A. visuelle/optische Messungen wie die PPG-Sensoren.

\subsection{Etablierte Methoden}
\label{etablierte methoden}

\subsubsection{Klinische Anamnese}
\label{klinische anamnese}

Klinische Anamnese ist das derzeit am häufigsten verwendete Verfahren zur Beurteilung des Hydratationszustandes.
Dabei werden Symptome betrachtet, die vom Patienten wahrgenommen oder vom praktizierenden Arzt erkannt werden.
 
Die einfachste Maßnahme besteht darin, den Durst einer Person zu beurteilen, indem man Methoden wie eine visuelle Analogskala oder eine kategorische Skala verwendet, mit der die Person ihren Durstgrad angibt.
Das Durstgefühl spiegelt jedoch nicht ausreichend den Wasserbedarf in besonders dehydriergefährdeten Personengruppen wie älteren Erwachsenen wider \cite{garret2018engineering}.
Eine Kombination aus dem Grad des morgendlichen Durstgefühls und dem Urinvolumen wurde von Armstrong et al. als Methode zur Identifizierung einer leichten Dehydratation vorgeschlagen \cite{armstrong2013novel}.

Ein weiterer sehr einfacher Test ist der Hautturgortest oder Kneiftest. 
Der Hautturgor ist ein Maß dafür, wie widerstandsfähig die Haut einer Person gegen Veränderungen ihrer Form ist.
Dafür wird die Haut auf dem Handrücken in Zeltform eingeklemmt und für einige Sekunden gehalten.
Wenn eine Person genügend Flüssigkeit in ihrem Körper hat, kehrt die Haut schnell in den vorherigen Zustand zurück.
Kehrt die haut nur sehr langsam in den Zustand vor dem Einklemmen zurück, ist das ein Zeichen für eine leichte Dehydratation.
Jedoch ist diese Methode sehr ungenau \cite{suryadevara2015towards}.

Ein niedriger systolischer Blutdruck hat eine hohe Spezifität bei der Diagnose hypotoner Dehydratation, aber eine schlechte Sensitivität \cite{fortes2015elderly}.
Weiterhin kann Dehydratation zu einer erhöhten Herzfrequenz in Ruhe und während leichten sportlichen Betätigung führen.
Zudem wurde die orthostatische Dysregulation als Indikator für den Hydratationsstatus untersucht. 
Dabei kommt es zu einem Blutdruckabfall beim Aufstehen aus der sitzenden oder liegenden Position.
Tatsächlich können Blutdruck- und Herzfrequenzreaktionen auf schnelle Veränderungen der Körperhaltung verwendet werden, um zusätzliche Informationen bei der Beurteilung des Hydratationszustandes zu liefern.
Diese klinischen Anzeichen können aber durch eine Reihe anderer Erkrankungen verursacht werden, wodurch sie zur Erkennung einer Dehydratation unbrauchbar sind, wenn sie unabhängig von anderen Indizes betrachtet werden
\cite{garret2018engineering}
\cite{kavouras2002assessing}
\cite{davis1997effect}.
Obwohl klinische Symptome relativ einfach zu beurteilen sind, sind sie im Allgemeinen nicht ausreichend sensitiv und spezifisch.

\subsubsection{Laboruntersuchungen}
\label{laboruntersuchungen}

\paragraph{Urinanalysen} sind relativ einfach durchzuführen und können eine schnelle Beurteilung der Hydratation ermöglichen.
Es gibt vier Hauptindizes zur Beurteilung der Hydratation: Farbe, Osmolalität, das spezifische Gewicht des Urins und Leitfähigkeit.

Urinanalysen haben mehrere Einschränkungen.
Sie werden als verzögerte Blutanalysen betrachtet.
Zusätzlich spiegeln die Ergebnisse die Eigenschaften des gesamten Urins wieder, der sich in der Blase angesammelt hat.
Eine kontinuierliche Überwachung ist aufgrund des relativ seltenen Urinierens unpraktisch.
Urinanalysen stellen daher wenig Nutzen für die Bewertung akuter Veränderungen der Hydratation dar.
Eine höhere Genauigkeit wird erreicht, wenn sie in einem stationären Zustand durchgeführt werden, wo sie Einblick in längere Hydratationsänderungen bieten können.

Kommt es zu einer Dehydratation, wird der Urin konzentriert und somit dunkler.
Armstrong et al.\cite{armstrong1994urinary} haben eine Sechs-Punkte-Likert-Skala zur Überwachung von der Urin-Farbe entwickelt.
Eine Kopie der Farbskala im Taschenformat bietet eine einfache, kostengünstige Methode zur Beurteilung der Dehydratation.
Urinverfärbungen können jedoch auch auf andere Faktoren zurückzuführen sein, wie z.B. Farbstoffe aus der Nahrung, bestimmte Medikamente, das Vorhandensein von Blut im Urin und Gelbsucht\cite{garret2018engineering}.
Dadurch ist die Messung der Hydratation mit einer Farbskala sehr ungenau.

Zur bestimmung des Osmolalität wird ein ausgebildeter Techniker und ein Gefrierpunktosmometer benötigt.
Das Osmometer misst die Stoffmenge gelöster Partikel pro Kilogramm Lösung. 
Es werden nur gelöste Stoffe, die dissoziieren, wie z.B. NaCl, nicht aber Partikel wie Glucose, Harnstoff und Proteine nachgewiesen \cite{oppliger2002hydration}. 
Das Ermitteln von Basiswerten für jedes Individuum ist notwendig, um die Hydratation mit dieser Methode richtig zu bestimmen, da die durchschnittliche Osmolalität in gesunden Personen stark zwischen den unterschiedlichen Kulturen variiert \cite{garret2018engineering}.

Das spezifische Gewicht des Urins (SGU) ist die Dichte einer Urinprobe im Vergleich zur Dichte von Wasser.
Das spezifische Gewicht der Probe ist abhängig von ihrer Osmolalität sowie ihrer Konzentration an Harnstoff, Glukose und Protein. 
Es gibt mehrere kostengünstige Methoden zur Überwachung des SGU, darunter Hygrometrie, Refraktometrie und Teststreifen. 
In der Hygrometrie wird ein gewichteter Glaskörper verwendet, um die Dichte des Urins im Vergleich zu reinem Wasser zu bestimmen.
Refraktometrie bedeutet, dass ein Lichtstrahl durch eine Urinprobe geleitet wird und gemessen wird, wie stark der Strahl gebrochen wird.
Die Brechung hängt von der Temperatur und Konzentration des Urins ab.
Teststreifen bieten eine einfache Alternative zur Messung des SGU im Vergleich zur weit verbreiteten Refraktometriemethode.
Zur Einschätzung des SGU erfolgt ein Vergleich mit einer Farbkarte.
Jede dieser Methoden erfordert allerdings eine gewisse technische Kompetenz, die jedoch von Klinikern oder Sporttrainern leicht erlernt werden kann \cite{oppliger2002hydration}.

Die elektrische Leitfähigkeit des Urins hängt mit der Osmolarität zusammen und wurde als Methode zur Beurteilung des Hydratationszustandes vorgestellt.
Die Leitfähigkeit ist eine Funktion der Gesamtkonzentration der Ionen, die in einer Probe enthalten sind und mit tragbaren Geräten  gemessen werden kann \cite{shirreffs1998urine}.
Das Gerät legt eine kleine Spannung an und misst den entsprechenden elektrischen Strom in der Urinprobe.
Es liefert dann Messwerte von 1 bis 5, die den Leitfähigkeitsbereichen entsprechen \cite{garret2018engineering}.

\paragraph{Blutanalyse} kann zur Messung einer Vielzahl von Parametern verwendet werden, die im Zusammenhang mit der Dehydratation stehen.
Dazu zählen die Plasma-Osmolalität (Posm), Elektrolyte, Blut-Harnstoff-Stickstoff (englisch blood urea nitrogen, BUN) zu Kreatinin (Cr) Verhältnis (BUN:Cr) und Hämoglobin/Hämatokritspiegel \cite{oppliger2002hydration}.
Blutwerte eignen sich aber nur zur Messung bei Veränderungen des Gesamtkörperwassers, die größer sind als 3\% des Körpergewichts \cite{francesconi1987urinary}.
Das bedeutet, dass sie weniger empfindlich sind als Urinanalysen.
Dies ist wahrscheinlich, weil der Körper versucht die normale Blutchemie so lange wie möglich aufrechtzuerhalten.
Außerdem sind die meisten Blutwerte nur für die hypertone Dehydratation anwendbar, da andere Formen der Dehydratation die Blutzusammensetzung nicht verändern \cite{garret2018engineering}.

Der aktuelle klinische Standard der Hydratationsmessung ist Plasma-Osmolalität.
Blut besteht sowohl aus Plasma als auch aus Blutzellen.
Posm wird mit einem Osmometer unter Verwendung von Techniken wie der Gefrierpunktdepression gemessen.
Werte für Posm variieren von 280 bis 290 mOsm/kg für eine normale Person.
Werte >290 mOsm/kg zeigen eine Dehydratation an.
Obwohl es oft als Referenzstandard für die Hydratation verwendet wird, gibt es mehrere Einschränkungen von Posm.
Diese Messungen sind relativ kostspielig im Vergleich zu anderen Ansätzen.
Die Tests benötigen weiterhin viel Zeit, weil eine Laboranalyse durchgeführt werden muss, und sie erfordern eine Venenpunktion, die Schmerzen oder Blutergüsse verursachen kann \cite{armstrong1994urinary}. 
Dies macht Blutanalysen unpraktisch für häufige, tragbare Anwendungen, und/oder zeitkritische Nutzung. 
Während Posm ein wichtiger Standard bei der Beurteilung der Genauigkeit anderer Ansätze ist, deuten diese Feststellungen darauf hin, dass es möglicherweise nicht der ideale Indikator für den Hydratationsstatus ist\cite{garret2018engineering}.

Zur Beurteilung der hypotonen Dehydratation kann das BUN:Cr-Verhältnis verwendet werden. 
Ein normales Verhältnis ist 40-100:1, gemessen mit dem Internationalen System der Einheiten(SI).
Hypotone Dehydratation ist definiert als BUN:Cr >100:1 mit SI-Einheiten, wenn keine Hypertonie vorliegt.
Um diesen Test durchzuführen, muss eine Vollblutprobe entnommen und innerhalb von 2 Stunden nach der Entnahme analysiert werden.
BUN:Cr kann jedoch auch aus anderen Gründen als Dehydratation erhöht sein, wie z.B. einer gastrointestinalen Blutung \cite{garret2018engineering}.

Des Weiteren können Hämoglobin (Hb) und Hämatokrit (Hct) Spiegel gemessen werden.
Dies erfordert nur einen Tropfen Blut und zur Analyse der Probe kann eine tragbare Maschine verwendet werden. 
Die relative Veränderung zwischen unterschiedlichen Zeitpunkten in einem oder beiden dieser Indizes können verwendet werden um Veränderungen im Blut-, Plasma- und Zellvolumen einzuschätzen.
Bei akuter Dehydratation steigt die Konzentration von Hb und Hct, da der Wassergehalt verloren geht, aber rote Blutkörperchen im Blut bleiben \cite{garret2018engineering}.
Die Einfachheit des Tests und die begrenzte Menge an benötigtem Blut machen ihn für die Praxis attraktiver als andere Blutwerte wie Posm.

Zusammenfassend lässt sich sagen, dass Blutanalysen genaue und präzise Methoden zur Beurteilung des Hydratationszustandes liefern können, aber aufgrund ihrer inhärenten Invasivität nicht für eine kontinuierliche Überwachung geeignet sind.
Blutwerte eignen sich daher am besten für die einmalige klinische Beurteilung.

\paragraph{Messung des Gewichts} ist eine einfache Methode zur Bestimmung von Veränderungen des Gesamtkörperwassers.
Über kurze Zeiträume hinweg ist diese Methode genau, wenn keine Lebensmittel oder Getränke konsumiert werden, es zu keiner Entleerung der Blase kommt, oder diese Faktoren berücksichtigt werden, und wenn mit größter Sorgfalt beim Wiegen der Person vorgegangen wird \cite{garret2018engineering}.
Darüber hinaus erfordert die Bewertung des Hydratationszustandes mit dieser Technik Vorkenntnisse über das Basisgewicht vor der Dehydratation.
Veränderungen des Körpergewichts können zur Beurteilung der Dehydratation in Kombination mit anderen Indikatoren herangezogen werden, insbesondere bei Menschen, die keine signifikanten Schwankungen ihres Körpergewichts erleiden. 
Trotz dieser Einschränkungen bleibt die Messung des Hydratationszustandes durch Veränderungen des Körpergewichts der einzige quantitative Ansatz, der sicher im Labor/Experimentalbereich eingesetzt werden kann \cite{kavouras2002assessing}.

\paragraph{Bioimpedanzanalyse} (BIA) gilt als nicht-invasives, schnelles und zuverlässiges Verfahren um das Gesamtkörperwasser zu bestimmen \cite{kavouras2002assessing}.
Jedoch sind solche Messungen insbesondere außerhalb idealer Laborbedingungen ungenau.
Schweiß, Hauttemperatur, Elektrodenplatzierung und Körperhaltung haben einen Einfluss auf die Messung.
Höchste Genauigkeit wird erst unter standardisierten Bedingungen erreicht, wie z.B. die Reinigung der Haut mit Alkohol vor dem Platzieren der Elektroden, genaue Bestimmung von Körpergröße und Gewicht, das Fasten für 4 Stunden vor der Messung und Vermeidung von sportlicher Aktivität. 
Obwohl BIA als potenzielle Technik zur Beurteilung des Hydratationszustandes vielversprechend ist, sind weitere Untersuchungen erforderlich, bevor es als diagnostisches Instrument zur Untersuchung von Veränderungen des Körperwassers eingesetzt werden kann \cite{garret2018engineering} \cite{kavouras2002assessing}.


\subsection{Aufkommende Methoden}
\label{aufkommende methoden}


\section{Tragbare Photoplethysmographie Sensoren}
\label{tragbare photoplethysmographie sensoren}


\subsection{Verwendung von Photopletysmographie zur physiologischen Messung}
\label{verwendung von photopletysmographie zur physiologischen messung}


\subsection{Verarbeitung des PPG-Signals}
\label{verarbeitung des ppg-signals}

\section{Vergleich der Zuverlässigkeit der PPG-Sensoren mit klinischen Methoden}
\label{vergleich der verfahren}

\section{Schlussbetrachtung und Ausblick}
\label{schlussbetrachtung und ausblick}


%
%%%%%%%%%%%%%%%%%%%%%%%%%%%%%%%%%%%%%%%%%%%%%%%%%%%%%%%%%%%%%%%%%%%%%%%%%%%%%%%%%%%%
% Literaturverzeichnis
\printbibliography
% Anhang
%\appendix
%\section{Anhang}
%\label{anhang}
\end{document}

