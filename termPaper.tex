\documentclass[10pt,a4paper,headinclude,twoside, plainheadsepline, open=right, numbers=noenddot, twocolumn]{article}
%
% ------
% Maketitle metadata
\title{\vspace{-5mm}%
	\fontsize{20pt}{10pt}\selectfont
	\textbf{Methoden zur Messung der Hydratation durch tragbare Photoplethysmographie-Sensoren}
	}	
\vspace{-5mm}\date{}
\author{
	\large
       \begin{minipage}[t]{0.33\linewidth}
         \begin{center}
           	\textsc{Olga Litau}\\[2mm]
                 \normalsize	Matr.Nr: 3156218\\
                 \normalsize
                 \href{mailto:olga1.litau@st.oth-regensburg.de}
                 {olga1.litau@st.oth-regensburg.de}      
         \end{center}
       \end{minipage}        
     }
%
%\addto\captionsgerman{\renewcommand{\figurename}{Fig.}}
%
%%%%%%%%%%%%%%%%%%%%%%%%%%%%%%%%%%%%%%%%%%%%%%%%%%%%%%%%%%%
%
% Literaturverzeichnis
%
% Hier eine von zwei Varianten auswaehlen:
% Nummern oder Buchstaben fuer Referenzen
%\usepackage[backend=biber, style=alphabetic, sorting=nyt]{biblatex}
\usepackage[backend=biber, style=numeric-comp, sorting=none]{biblatex}
%
% Hier werden die Referenzen in einer separaten Datei gespeichert
\addbibresource{termPaper.bib}
%
% WICHTIG: Hier wird nicht BibTeX sondern BibLateX verwendet!!
% Deshalb nicht mit bibtex uebersetzen, sondern mit biber
% Das kann man in jedem Tool wie TexMaker oder TexShop als Option einstellen
%

% Spezielle Einstellungen, insbesondere fuer das Literaturverzeichnis,
% aber auch Packages wie amsmath, Groessenanpassungen etc.
\input{./preferences.tex}
%
% ------
% Header/footer
\usepackage{fancyhdr}
	\pagestyle{fancy}
	\fancyfoot[C]{Wissenschaftliches Seminar WS 2018/19 $\cdot$
          $\cdot$ Prof.~Dr.~Doering}
%
	\fancyhead[RE]{Olga Litau}
	\fancyhead[LO]{Methoden zur Messung der Hydratation durch tragbare Photoplethysmographie
Sensoren}	
	\fancyhead[RO,LE]{\thepage}
%
\begin{document}
\pagenumbering{arabic} % ab jetzt arabische Nummerierung
\twocolumn[
%%%%%%%%%%%%%%%%%%%%%%%%%%%%%%%%%%%%%%%%%%%%%%%%%%%%%%%%%%%%%%%%%%%%%
\maketitle
\tableofcontents % Inhaltsverzeichnis
\vspace{2cm}
\begin{abstract}
\noindent Hierher kommt die Zusammenfassung...
\end{abstract}
\vspace{0.2cm}
]


\section{Einleitung}
\label{einleitung}
Dehydratation bezeichnet eine übermäßige Abnahme des Körperwassers, die eine normale tägliche Schwankung überschreitet. 
Mit 50-70\% der Gesamtkörpermasse ist Wasser der chemische Hauptbestandteil des menschlichen Körpers.
Für einen durchschnittlichen jungen Mann mit 70kg Körpergewicht bedeutet das ein Gesamtkörperwasser von 42l \cite{sawka2015hypohydration}.
5-10\% des Gesamtkörperwassers werden täglich umgesetzt \cite{raman2004american}.
Ein Körperwasserdefizit entsteht wenn die Flüssigkeitsaufnahme durch Getränke und Lebensmittel geringer ist als die Ausscheidung durch Urin, Schweiß, Kot und Verdunstung aus den Atemwegen und der Haut \cite{garret2018engineering}.
Der Begriff "Dehydratation" wird häufig als Synonym für hypertone Dehydratation verwendet.
Unter hypertoner Dehydratation versteht man einen Verlust von Wasser ohne entsprechenden Salzverlust. 
Verursacht werden kann eine hypertone Dehydratation durch unzureichende Flüssigkeitsaufnahme, Schwitzen oder Erbrechen.
Diese Form der Dehydratation ist zu unterscheiden von istoner Dehydratation und hypotoner Dehydratation.
Isotone Dehydratation kennzeichnet ein Verlust von Wasser und Salz-Ionen im gleichen Verhältnis.
Eine isotone Dehydratation tritt beispielsweise als Folge von Durchfall auf.
Hypotone Dehydratation entsteht, wenn im Verhältnis zum Wasserverlust zu viel Salz ausgeschieden wird.
Bei der Verwendung eines Diuretikums, das die vermehrte Ausschwemmung von Urin aus dem menschlichen Körper bewirkt, kann eine hypotone Dehydratation verursacht werden. 
Bei einer Messung ist es von Vorteil zwischen den Arten der Dehydrierung differenzieren zu können, um eine effektive Behandlung zu ermöglichen \cite{garret2018engineering}.

Die Methoden zur Messung der Dehydratation sind unterschiedlich und reichen von der klinischen Anamnese über Untersuchungen im Labor bis hin zu tragbaren Sensoren.
Kapitel \ref{methoden zur messung der hydratation} wird einen Überblick über die verschiedenen Methoden geben. 

Vor kurzem wurde von \textbf{Saryadevara et al.} die Verwendung von tragbaren Photoplethysmographie-Sensoren (PPG) zur objektiven Quantifizierung des Hydratationsstatus vorgeschlagen \cite{suryadevara2015towards}.
Kaptel \ref{tragbare photoplethysmographie sensoren} richtet den Fokus auf tragbare PPG-Sensoren und erläutert sowohl die Verwendung von Photopletysmographie zur physiologischen Messung als auch die Verarbeitung eines PPG-Signals.

In Kapitel \ref{vergleich der verfahren} wird die Zuverlässigkeit von PPG-Methoden anhand der verfügbaren Daten im Vergleich zu klinischen Methoden bewertet.

Abschließend erfolgt in Kapitel \ref{schlussbetrachtung und ausblick} eine Schlussbetrachtung mit Ausblick. 



\section{Methoden zur Messung der Hydratation}
\label{methoden zur messung der hydratation}


\subsection{Klinische Anamnese}
\label{klinische anamnese}


\subsection{Laboruntersuchungen}
\label{laboruntersuchungen}


\subsection{Tragbare Sensoren}
\label{tragbare sensoren}


\section{Tragbare Photoplethysmographie Sensoren}
\label{tragbare photoplethysmographie sensoren}


\subsection{Verwendung von Photopletysmographie zur physiologischen Messung}
\label{verwendung von photopletysmographie zur physiologischen messung}


\subsection{Verarbeitung des PPG-Signals}
\label{verarbeitung des ppg-signals}

\section{Vergleich der Zuverlässigkeit der PPG-Sensoren mit klinischen Methoden}
\label{vergleich der verfahren}

\section{Schlussbetrachtung und Ausblick}
\label{schlussbetrachtung und ausblick}


%
%%%%%%%%%%%%%%%%%%%%%%%%%%%%%%%%%%%%%%%%%%%%%%%%%%%%%%%%%%%%%%%%%%%%%%%%%%%%%%%%%%%%
% Literaturverzeichnis
\printbibliography
% Anhang
%\appendix
%\section{Anhang}
%\label{anhang}
\end{document}

